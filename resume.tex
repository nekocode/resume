%-------------------------
% Resume in Latex
% Author : nekocode, Sourabh Bajaj
% License : MIT
%-------------------------

\documentclass[letterpaper,11pt]{article}

\usepackage{latexsym}
\usepackage[empty]{fullpage}
\usepackage{titlesec}
\usepackage{marvosym}
\usepackage[usenames,dvipsnames]{color}
\usepackage{verbatim}
\usepackage{enumitem}
\usepackage{hyperref}
\usepackage{fancyhdr}

% Set chinese font
\usepackage{fontspec}
\usepackage{xeCJK}
\setCJKmainfont{STKaiti}
\linespread{1.2}\selectfont
\XeTeXlinebreaklocale "zh"
\XeTeXlinebreakskip = 0pt plus 1pt
\usepackage{xltxtra,xunicode}

\pagestyle{fancy}
\fancyhf{} % clear all header and footer fields
\fancyfoot{}
\renewcommand{\headrulewidth}{0pt}
\renewcommand{\footrulewidth}{0pt}

% Adjust margins
\addtolength{\oddsidemargin}{-0.5in}
\addtolength{\evensidemargin}{-0.5in}
\addtolength{\textwidth}{1in}
\addtolength{\topmargin}{-.5in}
\addtolength{\textheight}{1.0in}

\urlstyle{same}

\raggedbottom
\raggedright
\setlength{\tabcolsep}{0in}

% Sections formatting
\titleformat{\section}{
  \vspace{-4pt}\scshape\raggedright\large
}{}{0em}{}[\color{black}\titlerule \vspace{-5pt}]

% https://tex.stackexchange.com/questions/99316/symbol-for-external-links
\usepackage{tikz}
\newcommand{\ExternalLink}{%
    \tikz[x=1.0ex, y=1.0ex, baseline=-0.05ex]{% 
        \begin{scope}[x=1ex, y=1ex]
            \clip (-0.1,-0.1) 
                --++ (-0, 1.2) 
                --++ (0.6, 0) 
                --++ (0, -0.6) 
                --++ (0.6, 0) 
                --++ (0, -1);
            \path[draw, 
                line width = 0.5, 
                rounded corners=0.5] 
                (0,0) rectangle (0.8,0.8);
        \end{scope}
        \path[draw, line width = 0.5] (0.5, 0.5) 
            -- (1, 1);
        \path[draw, line width = 0.5] (0.6, 1) 
            -- (1, 1) -- (1, 0.6);
        }
    }

%------------Custom commands-------------
\newcommand{\resumeItem}[2]{
  \item\small{
    {#1}{: #2 \vspace{-2pt}}
  }
}

\newcommand{\resumeSubheading}[3]{
  \vspace{-1pt}\item
    \begin{tabular*}{0.97\textwidth}{l@{\extracolsep{\fill}}r}
      {#1} & \textit{\small #2} \\
      {\small #3} \\
    \end{tabular*}\vspace{-5pt}
}

\newcommand{\resumeSubItem}[2]{\resumeItem{#1}{#2}\vspace{-4pt}}
\newcommand{\resumeSubHeadingListStart}{\begin{itemize}[leftmargin=*]}
\newcommand{\resumeSubHeadingListEnd}{\end{itemize}}
\newcommand{\resumeItemListStart}{\begin{itemize}}
\newcommand{\resumeItemListEnd}{\end{itemize}\vspace{-5pt}}

\newcommand{\link}[2]{\href{#1}{#2}\ExternalLink}



%-------------------------
% Document starts here
%-------------------------

\begin{document}

%------------HEADING-------------
\begin{tabular*}{\textwidth}{l@{\extracolsep{\fill}}r}
  \textbf{\Huge 杨凡} & Email : \link{mailto:nekocode.cn@gmail.com}{nekocode.cn@gmail.com}\\
  \link{http://nekocode.cn}{nekocode.cn} & Mobile : +183-1092-7892 \\
\end{tabular*}


%------------EDUCATION-------------
\section{\textbf{Education}}
  \resumeSubHeadingListStart
    \resumeSubheading
      {广州大学}{2011/09 - 2015/06}
      {软件工程,本科学士,CET-4}
  \resumeSubHeadingListEnd


%------------EXPERIENCE-------------
\section{\textbf{Experience}}
  \resumeSubHeadingListStart

    \resumeSubheading
      {智者四海(北京)技术有限公司}{2017/02 - 至今}
      {Android Engineer}
      \resumeItemListStart
        \resumeItem{知乎 Android Client(千万级日活)}
          {开发知识市场相关业务,并负责对应用性能进行 Profile \& 优化。}
        \resumeItem{\link{https://github.com/zhihu/zhihu-rxjava-meetup}{「知乎 x RxJava Meetup」}}
          {作为组内 ReactiveX 技术核心推广者,对外分享 RxJava 在知乎内的推广和应用。}
        \resumeItem{\link{https://github.com/nekocode/ResourceInspector}{ResourceInspector}}
          {依赖 Stetho 开发的 Debug Tool,用于快速查看当前页面所使用 Layout 资源。}
      \resumeItemListEnd

    \resumeSubheading
      {快乐迭代(北京)网络科技有限公司}{2016/10 - 2017/01}
      {Android Tech Lead}
      \resumeItemListStart
        \resumeItem{滑滑 Android Client}
          {指导团队使用 Kotlin 语言重构整个客户端,并在组内普及单元测试。}
        \resumeItem{\link{https://github.com/nekocode/Kotlin-Android-Template}{Kotlin-Android-Template}}
          {使用 Kotlin/MVP/ReactiveX 搭建的项目模板,用于为公司快速开发新项目。}
        \resumeSubItem{\link{https://github.com/nekocode/ItemPool}{ItemPool} (Java)}
          {用于对 Android RecyclerView 里的 ViewHolder 和 Adapter 做解耦。}
        \resumeItem{Maven, CI Services}
          {使用 Docker 为公司搭建 Maven 以及持续集成服务,维护了 \link{https://github.com/nekocode/docker-android}{docker-android} 以及自动打包等镜像。}
      \resumeItemListEnd

    \resumeSubheading
      {广东数库互联网金融信息服务有限公司}{2015/01 - 2016/09}
      {Android Tech Lead \& Python Enigneer}
      \resumeItemListStart
        \resumeItem{数库金服 Android Client}
          {使用 Kotlin/ReactiveX 技术栈搭建。}
        \resumeItem{\link{https://github.com/nekocode/weixin_vote}{公众号投票平台},\link{https://github.com/nekocode/wcmovie_test}{测试小游戏平台}}
          {使用 Python 语言为公司开发各种公众号后端。}
      \resumeItemListEnd

  \resumeSubHeadingListEnd


%------------SIDE PROJECTS-------------
\section{\textbf{Side Projects (在 \href{https://github.com/nekocode}{github.com/nekocode} 上查看更多)}}
  \resumeSubHeadingListStart
    \resumeSubItem{\link{https://github.com/nekocode/CameraFilter}{CameraFilter} (Java)}
      {把 Shadertoy.com 上多个滤镜的 Shader 代码移植到 Android 系统上。}
    \resumeSubItem{\link{https://github.com/nekocode/Hubs}{Hubs} (Java/Lua)}
      {内容 Hub 应用,支持使用 Lua 语言开发插件。}
    \resumeSubItem{\link{https://github.com/nekocode/Badge}{Badge} (Java)}
      {Android 系统下的 Badge Drawable。}
    \resumeSubItem{\link{https://github.com/nekocode/Murmur}{Murmur} (Kotlin/RxJava)}
      {第三方豆瓣 FM 红心频道播放器,使用 Kotlin 和 Java 语言开发。}
    \resumeSubItem{\link{https://github.com/nekocode/Meepo}{Meepo} (Java)}
      {Android 系统下的通用路由生成器。}
    \resumeSubItem{\link{https://github.com/nekocode/doubanfm-py}{doubanfm-py} (Python)}
      {使用 Python 语言开发的 Terminal 版豆瓣电台。}
    \resumeSubItem{\link{https://github.com/nekocode/android-parcelable-intellij-plugin-kotlin}{android-parcelable-intellij-plugin-kotlin} (Java)}
      {Intellij 插件,可以为 Kotlin 类生成 Parcelable 模板代码,目前有近 \link{https://plugins.jetbrains.com/plugin/8086-parcelable-code-generator-for-kotlin-}{3 万下载量。}}
    \resumeSubItem{\link{https://github.com/nekocode/BattleBeats}{BattleBeats} (C++)}
      {使用 C++/DirectX 开发的弹幕游戏,获院游戏制作比赛一等奖。}
  \resumeSubHeadingListEnd


\end{document}